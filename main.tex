\documentclass[10pt,openany,onesided]{book}
\usepackage{array}
\usepackage{booktabs}
\usepackage[utf8]{inputenc}
\usepackage{amsmath, amssymb}
\usepackage{geometry}
\usepackage{hyperref}
\usepackage{graphicx}
\usepackage{fancyhdr}
\usepackage{hyperref}
\usepackage{float}
\usepackage{subfig} % Added for subfloat functionality
\usepackage{subfiles}
% Page layout
\geometry{a4paper, margin=1in}

% Header and footer
\pagestyle{fancy}
\fancyhf{}
\fancyhead[L]{Machine Learning}
\fancyhead[R]{\thepage}
\setlength{\headheight}{14.49998pt} % Adjust head height
\addtolength{\topmargin}{-2.49998pt} % Compensate for the increased head height
\hypersetup{
    colorlinks,
    linkcolor=black,
    filecolor=black,
    urlcolor=cyan,
}
\newtheorem{theorem}{Theorem}
\title{Appunti di machine learning}
\author{Daniele Besozzi}
\date{Anno accademico 2025/2026}
\begin{document}
\frontmatter
\maketitle	
\tableofcontents
\newpage
\mainmatter
\chapter*{Premesse}
Questi sono appunti realizzati per riassumere e schematizzare tutti i concetti presentati durante il corso di machine learning tenuto presso il corso di laurea magistrale in informatica presso l'università degli studi di Milano Bicocca.
Lo scopo di questo documento non è quello di sostituire le lezioni del corso o di essere l'unica fonte di studio, bensì integrare le altri fonti con un documento riassuntivo.
\\
Mi scuso in anticipo per eventuali errori e prego i lettori di segnalarli contattandomi via mail all'indirizzo \href{mailto:d.besozzi@campus.unimib.it}{d.besozzi@campus.unimib.it}.
\subfile{sections/introduzione.tex}
\subfile{sections/ConceptLearning.tex}
\subfile{sections/FeatureEngineering.tex}
\subfile{sections/DecisionTrees.tex}
\end{document}